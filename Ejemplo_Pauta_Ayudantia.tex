\documentclass[letterpaper,11pt,oneside]{article}
\usepackage{geometry} %diseño de páginas
\usepackage{tikz} %gráficos en latex
	\usetikzlibrary{datavisualization}
	\usetikzlibrary{datavisualization.formats.functions}
	\usetikzlibrary{shapes,arrows}
\usepackage{pgfplots} %gráficos de funciones
	\pgfplotsset{compat=1.3}
\usepackage[hidelinks]{hyperref} %para hipervínculos
\usepackage[spanish,es-noshorthands]{babel} %español con tikz
%\usepackage[spanish]{babel} %español
\usepackage[utf8]{inputenc} %input caracteres especiales
\usepackage{fontenc} %output caracteres especials
\usepackage{graphicx} %gráficas
	\graphicspath{{../imagenes/}} %ruta de imágenes
\usepackage{verbatim} %comentarios
\usepackage{setspace} %espaciado
\usepackage{lipsum} %texto de relleno para pruebas
\usepackage{wrapfig} %texto alrededor de flotantes
\usepackage{amsmath} %ambiente equation
\usepackage{amssymb} %símbolos
\usepackage{wasysym} %más símbolos
\usepackage{multicol} %múltiples columnas en tablas
\usepackage{adjustbox} %ajustar tablas
\usepackage{enumitem} %listas
\usepackage{array} %tablas en modo matemático
\usepackage{afterpage} %página en blanco
\usepackage{fancyhdr} %encabezado y pie de página
\usepackage{multirow} %múltiples filas en tablas

%CONFIGURACIÓN DE PÁGINA
\setlength{\topmargin}{-40pt}
\setlength{\headsep}{35pt}
\setlength{\textwidth}{460pt}
\setlength{\oddsidemargin}{7pt}
\setlength{\textheight}{640pt}
\setlength{\parindent}{0pt}
\setlength{\parskip}{5pt}
\setlength{\headheight}{25.3pt}

%CREACION COMANDO SIGNO DE GRADO
\newcommand{\grad}{$^{\circ}$}

%USAR BULLET POINT EN EL PRIMER ITEM
\renewcommand{\labelitemi}{$\bullet$}

%ENCABEZADO
\lhead{\textbf{Universidad Técnica Federico Santa María} \\ Departamento de Ingeniería Comercial}
\rhead{\begin{picture}(0,0) \put(-85,-6){\includegraphics[scale=0.19]{logo_usm}} \end{picture}}

%PIE DE PAGINA
\lfoot{2018-S2} %MODIFICAR SEGÚN CORRESPONDA
\rfoot{\thepage}
\cfoot{\ }

\pagestyle{fancy}

\begin{document}
	\begin{center}
	    %MODIFICAR SEGÚN CORRESPONDA
		{\Large Gestión de Investigación de Operaciones - ICS010}\\
		%MODIFICAR SEGÚN CORRESPONDA
		{\large Pauta Ayudantía N\grad 1 - Segundo Semestre 2018}
	\end{center}

	\begin{center}
	    %MODIFICAR SEGÚN CORRESPONDA
		Profesor: Humberto Villalobos Torres\\
		%MODIFICAR SEGÚN CORRESPONDA
		Ayudantes: Alexandra Gallardo Silva - Matías Schiaffino Tyrer
	\end{center}

\section*{Ejercicio 1: Problema de la Dieta}

Ozark Farms consume diariamente un mínimo de 800 $[lb]$ de un alimento especial, el cual es una mezcla de maíz y soya con las siguientes composiciones:

\begin{table}[htb]
	\centering
	\begin{tabular}{l c c c}
		\hline\hline
		& \multicolumn{2}{c}{$lb$ por $lb$ de forraje} & \\
		\cline{2-3}
		Forraje & Proteína & Fibra & Costo ($\$/lb$)\\
		\hline
		Maíz & .09 & .02 & .30\\
		Soya & .60 & .06 & .90\\
		\hline\hline
	\end{tabular}
\end{table}

Las necesidades dietéticas del alimento especial son un mínimo de 30\% de proteína y un máximo de 5\% de fibra. El objetivo es determinar la mezcla diaria de alimento a un costo mínimo.

\begin{itemize}
	\item[a)] Plantee un Problema de Programación Lineal, resuélvalo utilizando el Método Gráfico y clasifique las restricciones como activas, inactivas o redundantes.
	
	\begin{equation*}
	\begin{array}{c}
	x_{1}:\textup{libras de maíz en la mezcla diaria}\\
	x_{2}:\textup{libras de soya en la mezcla diaria}\\
	\end{array}
	\end{equation*}
	
	\begin{equation*}
	\begin{array}{l l}
	\multicolumn{2}{c}{FO:\ min\ W=0.3x_{1}+0.9x_{2}}\\
	s/a & \\
	x_{1}+x_{2}\geq 800 & (R_{1}-\textup{Requerimiento lb alimento especial})\\
	0.21x_{1}-0.30x_{2}\leq 0 & (R_{2}-\textup{Requerimiento proteína mezcla total})\\
	0.03x_{1}-0.01x_{2}\geq 0 & (R_{3}-\textup{Limitación fibra mezcla total})\\
	x_{j}\geq 0,\ \forall\ j \in \{1,2\} & (\textup{No negatividad})\\ 
	\end{array}
	\end{equation*}
	
	\renewcommand*{\arraystretch}{1.3}
	\begin{table}[hbt]
		\centering
		\begin{tabular}{c | c}
			\hline\hline
			Punto & Valor FO\\
			\hline
			\textbf{A}($\mathbf{\frac{8000}{17},\frac{5600}{17}}$) & $\mathbf{\frac{7440}{17}}$\\
			B(200,600) & 600\\
			\hline\hline
		\end{tabular}
	\end{table}
	
	\textbf{Respuesta:} La mezcla especial debería tener aproximadamente 470.6 $[lb]$ de maíz y 329.4 $[lb]$ de soya, lo que reporta un costo mínimo de \$437.6. Los requerimientos de mezcla total y proteína de la mezcla son restricciones activas, mientras que la limitación de fibra es inactiva.

	\newpage
	\begin{figure}[hbt]
		\centering
		\begin{tikzpicture}[scale=1.6]
		\begin{axis}[
		grid=both,
		axis lines=middle,
		axis line style={->},
		x label style={at={(axis cs:1080,-25)}},
		y label style={at={(axis cs:-20,1070)}},
		xmin=0, xmax=1000,
		ymin=0, ymax=1000,
		xtick distance=100,
		ytick distance=100,
		xlabel=$\scriptstyle x_{1}$,
		ylabel=$\scriptstyle x_{2}$,
		grid style={line width=.1pt, draw=gray!20},
		major grid style={line width=.2pt,draw=gray!60},
		minor tick num=4,
		ticklabel style={font=\tiny},
		]
		
		\addplot[
		domain=0:800,
		samples=100,
		color=black
		] {800 - x};
		
		\node at (axis cs: 100,740) {$\scriptscriptstyle R_{1}$};
		
		\addplot[
		domain=0:1000,
		samples=100,
		color=black
		] {0.7*x};
		
		\node at (axis cs: 240,210) {$\scriptscriptstyle R_{2}$};
		
		\addplot[
		domain=0:1000,
		samples=100,
		color=black
		] {3*x};
		
		\node at (axis cs: 100,400) {$\scriptscriptstyle R_{3}$};
		
		\addplot[
		fill=gray!15,
		fill opacity=0.8,
		] coordinates{
			(470.6,329.4)
			(1000,700)
			(1000,1000)
			(333.33,1000)
			(200,600)
			(470.6,329.4)
		};
		
		\addplot[
		mark=*,
		mark size=1pt,
		mark options={draw=black, fill=black}
		] coordinates{
			(470.6,329.4)
			(200,600)
		};
		
		\addplot[
		dashed,
		domain=0:1000,
		samples=100,
		color=black,
		] {486.27 - 0.33*x};
		
		\node at (axis cs: 470.6,300) {\tiny A};
		\node at (axis cs: 210,560) {\tiny B};
		\node at (axis cs: 800,250) {\tiny FO};
		\node at (axis cs: 600,740) {$\mathcal{F}$};
		
		\end{axis}
		\end{tikzpicture}
	\end{figure}
	
	\item[b)] ¿Cuánto pueden variar los costos del maíz y la soya de tal forma que se conserve la solución óptima actual?
	
	\begin{equation*}
	\begin{split}
	(1)&\ -1 \leq \frac{c_{1}}{c_{2}} \leq 0\\
	(2)&\ 0 \leq \frac{c_{1}}{c_{2}} \leq \frac{7}{10}\\
	\end{split}
	\end{equation*}
	
	\textbf{Costo maíz ($\mathbf{c_{1}}$)}
	
	\begin{equation*}
	\begin{split}
	(1.1)&\ -1 \leq -\frac{c_{1}}{0.9} \leq 0\ \Rightarrow\ 0 \leq c_{1} \leq 0.9\\
	(1.2)&\ 0 \leq -\frac{c_{1}}{0.9} \leq \frac{7}{10}\ \Rightarrow\ -0.63 \leq c_{1} \leq 0\\
	\end{split}
	\end{equation*}
	
	\textbf{Costo soya ($\mathbf{c_{2}}$)}
	
	\begin{equation*}
	\begin{split}
	(2.1)&\ -1 \leq -\frac{0.3}{c_{2}} \leq 0\ \Rightarrow\ -\infty \leq -\frac{c_{2}}{0.3} \leq -1\ \Rightarrow\ 0.3 \leq c_{2} \leq \infty\\
	(2.2)&\ 0 \leq -\frac{0.3}{c_{2}} \leq \frac{7}{10}\ \Rightarrow\ \frac{10}{7} \leq -\frac{c_{2}}{0.3} \leq \infty\ \Rightarrow\ -\infty \leq c_{2} \leq -\frac{3}{7}\\
	\end{split}
	\end{equation*}
	
	\newpage
	\begin{table}[hbt]
		\centering
		\begin{tabular}{c | c | c c}
			\hline\hline
			\multirow{2}{*}{Forraje}& \multirow{2}{*}{Intervalo} &\multicolumn{2}{c}{Permisible}\\
			\cline{3-4}
			& & Disminuir & Aumentar\\
			\hline
			Maíz & $[-0.63,0.9]$ & 0.93 & 0.6\\
			Soya & $(-\infty,-\frac{3}{7}] \cup [0.3,\infty)$ & $0.6 \vee (\infty,\frac{93}{70}]$ & $\infty$\\
			\hline\hline
		\end{tabular}
	\end{table}
	
	\textbf{Respuesta:} Para conservar el sentido económico del problema los costos no pueden adoptar valores negativos. Por lo tanto, el costo del maíz sólo puede variar entre $[0,0.9]$ y el de la soya entre $[0.3,\infty)$.
\end{itemize}

\section*{Ejercicio 2: Problema de Inversión}

Una persona tiene \$15.000 para invertir en dos tipos de acciones, A y B. El tipo A tiene una rentabilidad de 9\% anual, mientras que la de tipo B tiene 5\% anual. Esta persona decide invertir como máximo \$9.000 en acciones tipo A y un mínimo de \$3.000 en las tipo B. Además, le interesa invertir en A tanto o más que en B.

\begin{itemize}
	\item[a)] Plantee un Problema de Programación Lineal que permita maximizar la rentabilidad del inversionista y resuélvalo a través del Método Gráfico.
	
	\begin{equation*}
	\begin{array}{c}
	x_{1}:\textup{dinero invertido en acciones tipo A}\\
	x_{2}:\textup{dinero invertido en acciones tipo B}\\
	\end{array}
	\end{equation*}
	
	\begin{equation*}
	\begin{array}{l l}
	\multicolumn{2}{c}{FO:\ max\ Z=0.09x_{1}+0.05x_{2}}\\
	s/a & \\
	x_{1}\leq 9000 & (R_{1}-\textup{Limitación inversión tipo A})\\
	x_{2}\geq 3000 & (R_{2}-\textup{Requerimiento inversión tipo B})\\
	x_{1}-x_{2}\geq 0 & (R_{3}-\textup{Requerimiento inversión tipo A})\\
	x_{1}+x_{2}\leq 15000 & (R_{4}-\textup{Limitación presupuesto})\\
	x_{j}\geq 0,\ \forall\ j \in \{1,2\} & (\textup{No negatividad})\\ 
	\end{array}
	\end{equation*}
	
	\begin{table}[hbt]
		\centering
		\begin{tabular}{c | c}
			\hline\hline
			Punto & Valor FO\\
			\hline
			A(3000,3000) & 420\\
			B(9000,3000) & 960\\
			\textbf{C(9000,6000)} & \textbf{1110}\\
			D(7500,7500) & 1050\\
			\hline\hline
		\end{tabular}
	\end{table}
	
	\textbf{Respuesta:} El inversionista debería invertir \$9.000 en acciones tipo A y \$6.000 en tipo B obteniendo así una rentabilidad anual máxima de \$1.110.
	
	\newpage
	\begin{figure}[hbt]
		\centering
		\begin{tikzpicture}[scale=1.6]
		\begin{axis}[
		grid=both,
		axis lines=middle,
		axis line style={->},
		x label style={at={(axis cs:17300,-300)}},
		y label style={at={(axis cs:-300,17300)}},
		xmin=0, xmax=16000,
		ymin=0, ymax=16000,
		xtick distance=2000,
		ytick distance=2000,
		xlabel=$\scriptstyle x_{1}$,
		ylabel=$\scriptstyle x_{2}$,
		grid style={line width=.1pt, draw=gray!20},
		major grid style={line width=.2pt,draw=gray!60},
		minor tick num=4,
		scaled ticks=false,
		tick label style={font=\tiny},
		]
		
		\addplot[
		mark=none,
		black,
		line width=0.6pt,
		]coordinates{
		(9000,0)
		(9000,16000)
		};
		
		\node at (axis cs: 9400,14400) {$\scriptscriptstyle R_{1}$};
		
		\addplot[
		domain=0:16000,
		samples=100,
		color=black,
		] {3000};
		
		\node at (axis cs: 800,3400) {$\scriptscriptstyle R_{2}$};
		
		\addplot[
		domain=0:16000,
		samples=100,
		color=black,
		] {x};
		
		\node at (axis cs: 13750,14400) {$\scriptscriptstyle R_{3}$};
		
		\addplot[
		domain=0:16000,
		samples=100,
		color=black,
		] {15000 - x};
		
		\node at (axis cs: 2000,13600) {$\scriptscriptstyle R_{4}$};
		
		\addplot[
		fill=gray!15,
		fill opacity=0.8,
		] coordinates{
			(3000,3000)
			(9000,3000)
			(9000,6000)
			(7500,7500)
			(3000,3000)
		};
		
		\addplot[
		mark=*,
		mark size=1pt,
		mark options={draw=black, fill=black}
		] coordinates{
			(3000,3000)
			(9000,3000)
			(9000,6000)
			(7500,7500)
		};
		
		\addplot[
		dashed,
		domain=0:16000,
		samples=100,
		color=black,
		] {22200 - 1.8*x};
		
		\node at (axis cs: 3100,2600) {\tiny A};
		\node at (axis cs: 9300,2600) {\tiny B};
		\node at (axis cs: 9300,6200) {\tiny C};
		\node at (axis cs: 7540,7900) {\tiny D};
		\node at (axis cs: 12000,1600) {\tiny FO};
		\node at (axis cs: 7300,5000) {$\mathcal{F}$};
		
		\end{axis}
		\end{tikzpicture}
	\end{figure}
	
	\item[b)] Si es posible, determine el incremento en la rentabilidad anual por cada unidad monetaria adicional de presupuesto de tal forma que se conserve el contexto actual del problema.
	
	\begin{table}[hbt]
	    \centering
		\resizebox{0.9\textwidth}{!}{
		\begin{tabular}{c c c c c | c c}
			\hline\hline
			\multirow{2}{*}{Restricción} & \multirow{2}{*}{Valor Inicial} & \multirow{2}{*}{Punto} & \multirow{2}{*}{Valor Restricción} & \multirow{2}{*}{Valor FO} & \multicolumn{2}{c}{Permisible}\\
			\cline{6-7}
			& & & & & Disminuir & Aumentar\\
			\hline
			\multirow{2}{*}{$R_{4}$} & \multirow{2}{*}{15000} & max (9000,9000) & 18000 & 1260 & \multirow{2}{*}{3000} & \multirow{2}{*}{3000}\\
			& & min (9000,3000) & 12000 & 960 & & \\
			\hline\hline
		\end{tabular}}
	\end{table}
	
	\begin{equation*}
	\pi_{4}=\frac{FO(max)-FO(min)}{R_{4}(max)-R_{4}(min)}=\frac{1260-960}{18000-12000}=0.05
	\end{equation*}
	
	\textbf{Respuesta:} Como la limitación de presupuesto es una restricción activa, entonces se puede calcular su precio dual. Por cada unidad monetaria adicional de presupuesto, la rentabilidad incrementa en \$0.05, donde, además, este valor sólo es válido para presupuestos que se encuentren dentro del rango de $b_{4} \in [12000,18000]$.
\end{itemize}

\end{document}