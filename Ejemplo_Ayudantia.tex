\documentclass[letterpaper,11pt,oneside]{article}
\usepackage{geometry} %diseño de páginas
\usepackage[hidelinks]{hyperref} %para hipervínculos
\usepackage[spanish]{babel} %español
\usepackage[utf8]{inputenc} %input de caracteres especiales
\usepackage{fontenc} %output de caracteres especiales
\usepackage{graphicx} %gráficas
\usepackage{verbatim} %comentarios
\usepackage{setspace} %espaciado
\usepackage{lipsum} %texto de relleno para pruebas
\usepackage{wrapfig} %texto alrededor de flotantes
\usepackage{amsmath} %ambiente equation
\usepackage{amssymb} %símbolos matemáticos (extensión)
\usepackage{wasysym} %más símbolos
\usepackage{multicol} %múltiples columnas en tablas
\usepackage{adjustbox} %ajustar tablas
\usepackage{enumitem} %layout de listas
\usepackage{array} %tablas en modo matemático
\usepackage{afterpage} %página en blanco
\usepackage{fancyhdr} %pie y encabezado de página
\usepackage{multirow} %filas múltiples en tablas

%CONFIGURACIÓN DE PÁGINA
\setlength{\topmargin}{-40pt}
\setlength{\headsep}{35pt}
\setlength{\textwidth}{460pt}
\setlength{\oddsidemargin}{7pt}
\setlength{\textheight}{640pt}
\setlength{\parindent}{0pt}
\setlength{\parskip}{5pt}
\setlength{\headheight}{25.3pt}

%CREACIÓN COMANDO SIGNO DE GRADO
\newcommand{\grad}{$^{\circ}$}

%USAR BULLET POINT EN EL PRIMER ÍTEM DE LISTAS
\renewcommand{\labelitemi}{$\bullet$}

%ENCABEZADO
\lhead{\textbf{Universidad Técnica Federico Santa María} \\ Departamento de Ingeniería Comercial}
\rhead{\begin{picture}(0,0) \put(-85,-6){\includegraphics[scale=0.19]{logo_usm}} \end{picture}}

%PIE DE PAGINA
\lfoot{2018-S2} %MODIFICAR SEGÚN CORRESPONDA
\rfoot{\thepage}
\cfoot{\ }

%USAR ENCABEZADO Y PIE DE PÁGINA EN TODO EL DOCUMENTO
\pagestyle{fancy}

\begin{document}
	\begin{center}
	    %MODIFICAR SEGÚN CORRESPONDA
		{\Large Gestión de Investigación de Operaciones - ICS010}\\
		%MODIFICAR SEGÚN CORRESPONDA
		{\large Segundo Semestre - 2018}\\
		\ \\
		%MODIFICAR SEGÚN CORRESPONDA
		{\large Ayudantía N\grad 1}
	\end{center}

	\begin{center}
	    %MODIFICAR SEGÚN CORRESPONDA
		Profesor: Humberto Villalobos Torres\\
		%MODIFICAR SEGÚN CORRESPONDA
		Ayudantes: Alexandra Gallardo Silva - Matías Schiaffino Tyrer
	\end{center}

\section*{Ejercicio 1: Problema de la Dieta}

Ozark Farms consume diariamente un mínimo de 800 $[lb]$ de un alimento especial, el cual es una mezcla de maíz y soya con las siguientes composiciones:

\begin{table}[htb]
	\centering
	\begin{tabular}{l c c c}
		\hline\hline
		& \multicolumn{2}{c}{$lb$ por $lb$ de forraje} & \\
		\cline{2-3}
		Forraje & Proteína & Fibra & Costo ($\$/lb$)\\
		\hline
		Maíz & .09 & .02 & .30\\
		Soya & .60 & .06 & .90\\
		\hline\hline
	\end{tabular}
\end{table}

Las necesidades dietéticas del alimento especial son un mínimo de 30\% de proteína y un máximo de 5\% de fibra. El objetivo es determinar la mezcla diaria de alimento a un costo mínimo.

\begin{itemize}
	\item[a)] Plantee un Problema de Programación Lineal, resuélvalo utilizando el Método Gráfico y clasifique las restricciones como activas, inactivas o redundantes.
	\item[b)] ¿Cuánto pueden variar los costos del maíz y la soya de tal forma que se conserve la solución óptima actual?
\end{itemize}

\section*{Ejercicio 2: Problema de Inversión}

Una persona tiene \$15.000 para invertir en dos tipos de acciones, A y B. El tipo A tiene una rentabilidad de 9\% anual, mientras que la de tipo B tiene 5\% anual. Esta persona decide invertir como máximo \$9.000 en acciones tipo A y un mínimo de \$3.000 en las tipo B. Además, le interesa invertir en A tanto o más que en B.

\begin{itemize}
	\item[a)] Plantee un Problema de Programación Lineal que permita maximizar la rentabilidad del inversionista y resuélvalo a través del Método Gráfico.
	\item[b)] Si es posible, determine el incremento en la rentabilidad anual por cada unidad monetaria adicional de presupuesto de tal forma que se conserve el contexto actual del problema.
\end{itemize}

\end{document}